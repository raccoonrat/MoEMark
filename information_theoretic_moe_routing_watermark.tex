%%%%%%%%%%%%%%%%%%%%%%%%%%%%%%%%%%%%%%%%%%%%%%%%%%%%%%%%%%%%%%%%%%%%%%%%%%%%%%%%
% Template for USENIX papers.
% Based on usenix2020_SOUPS.sty template
% MoE路由水印的信息论形式化定义
%%%%%%%%%%%%%%%%%%%%%%%%%%%%%%%%%%%%%%%%%%%%%%%%%%%%%%%%%%%%%%%%%%%%%%%%%%%%%%%%

\documentclass[letterpaper,twocolumn,10pt]{article}
\usepackage{usenix2020_SOUPS}

% Chinese support packages
\usepackage{xeCJK}
\setCJKmainfont{SimSun}  % 使用宋体,如果没有可以改为其他中文字体
\setCJKsansfont{SimHei}  % 使用黑体
\setCJKmonofont{FangSong}  % 使用仿宋

% Additional packages
\usepackage{amsmath}
\usepackage{amssymb}
\usepackage{url}
\usepackage{hyperref}
\usepackage{booktabs}
\usepackage{array}

%-------------------------------------------------------------------------------
\begin{document}
%-------------------------------------------------------------------------------

%don't want date printed
\date{}

% make title bold and 14 pt font
\title{\Large \bf MoE路由水印的信息论形式化定义\\
\large Information-Theoretic Formalization of MoE Routing Watermarking}

%for single author (just remove % characters)
\author{
{\rm Yunhao}\\
% copy the following lines to add more authors
% \and
% {\rm Name}\\
%Name Institution
} % end author

\maketitle

%-------------------------------------------------------------------------------
\begin{abstract}
%-------------------------------------------------------------------------------
本文提出了混合专家(MoE)架构中路由水印的信息论形式化框架。与以往假设可直接控制路由器输出的方法不同,我们认识到路由器是一个习得的函数,只能通过修改模型参数来间接影响其行为。我们建立了基于参数扰动的信道模型:编码器将水印消息映射为模型参数的结构化扰动 $\Delta\theta$,信道是数据分布 $p(x)$ 对这些扰动的响应,输出是验证者在测试集上观测到的激活模式统计量。我们量化了激活模式(组合码)的编码容量,这是最鲁棒的信息维度。在约束模型性能和隐蔽性的条件下,我们建立了二维率失真(Rate-Distortion)框架,同时考虑性能失真 $D_{\text{perf}}$ 和统计失真 $D_{\text{detect}}$。我们提出了基于约束优化的编码器算法,通过带约束的微调过程实现水印嵌入。针对对抗性攻击,我们采用极小极大(min-max)准则定义鲁棒性,并基于 Le Cam/Pinsker 方法推导了可检性下界。在验证过程中,我们将问题形式化为复合假设检验,采用广义似然比检验(GLRT)或序贯概率比检验(SPRT)构建验证器,并基于 Chernoff/Bhattacharyya 上界间接推导了检测质量的 AUC 下界。最后,我们定义了综合性能指标和帕累托前沿,为 MoE 路由水印的设计和评估提供了严格的理论基础。
\end{abstract}


%-------------------------------------------------------------------------------
\section{引言}
%-------------------------------------------------------------------------------

混合专家(Mixture-of-Experts, MoE)架构通过稀疏激活机制实现了大型神经网络的高效设计。在 MoE 架构中,路由器(Router)负责为每个输入令牌选择性地激活专家子集,这种路由机制为水印嵌入提供了独特的信息载体。然而,目前缺乏对 MoE 路由水印的理论分析框架,特别是从信息论角度对其容量、鲁棒性和检测能力的严格量化。

\textbf{核心挑战}:以往的理论框架存在根本性缺陷,即假设可以直接控制路由器的输出(如路由分布 $\mathbf{r}$、激活集合 $\Sigma$ 等)。然而,在深度学习的现实中,路由器是一个习得的函数 $\mathbf{r} = \text{Router}(x)$,我们无法直接控制其输出,只能通过修改模型参数(如路由器的权重 $\mathbf{w}, b$)或训练过程来间接影响它。

本文提出了 MoE 路由水印的信息论形式化定义,旨在为路由水印的设计和评估提供理论基础。我们建立了基于参数扰动的信道模型,将编码器定义为从水印消息到模型参数扰动的映射,信道是数据分布对这些扰动的响应。我们量化了激活模式(组合码)的编码容量,这是最鲁棒的信息维度。在此基础上,我们建立了二维率失真框架,同时考虑性能失真和统计失真(隐蔽性),并提出了基于约束优化的编码器算法。本文的贡献包括:

\begin{itemize}
\item 建立了基于参数扰动的信道模型,明确了编码器、信道和输出的定义,避免了"直接控制输出"的错误假设
\item 量化了激活模式(组合码)的编码容量,这是最鲁棒的信息维度,而排列码和权重量化在攻击下几乎无法幸存
\item 建立了二维率失真(Rate-Distortion)框架,同时考虑性能失真 $D_{\text{perf}}$ 和统计失真 $D_{\text{detect}}$(隐蔽性)
\item 提出了基于约束优化的编码器算法,通过带约束的微调过程实现水印嵌入,使用基于间隔的水印损失函数
\item 采用极小极大(min-max)准则定义鲁棒性,并基于 Le Cam/Pinsker 方法推导了可检性下界
\item 构建了基于复合假设检验的验证框架(GLRT/SPRT),强调只使用激活模式特征进行验证
\end{itemize}

%-------------------------------------------------------------------------------
\section{基础符号与系统定义}
%-------------------------------------------------------------------------------

\subsection{MoE路由机制基础}

设 MoE 模型的第 $l$ 层为:
$$\text{MoE}_l(x) = \sum_{i=1}^{n} g_i(x) \cdot E_i(x)$$

其中:
\begin{itemize}
\item $x \in \mathbb{R}^d$:输入向量
\item $g_i(x)$:第 $i$ 个专家的路由权重(由路由器 Router 产生)
\item $E_i(x)$:第 $i$ 个专家的输出
\item $n$:专家总数
\end{itemize}

\textbf{路由器输出}:
$$\mathbf{r} = \text{Router}(x) = \text{Softmax}(\mathbf{w}^T x + b) \in \Delta^{n-1}$$

其中 $\text{Softmax}(\cdot)$ 为 softmax 函数,得到路由分布 $\mathbf{r} = [r_1, r_2, \ldots, r_n]$,位于概率单纯形 $\Delta^{n-1} = \{\mathbf{r} \in \mathbb{R}^n: r_i \geq 0, \sum_{i=1}^n r_i = 1\}$。温度参数 $T$ 用于控制分布的锐度:$\text{Softmax}_T(\mathbf{z})_i = \exp(z_i/T) / \sum_j \exp(z_j/T)$。

\subsection{水印系统定义}

水印系统 $\mathcal{W} = (\mathcal{E}, \mathcal{V})$ 包括:
\begin{itemize}
\item $\mathcal{E}$:嵌入过程(Embedding),将水印消息 $m$ 映射为模型参数的结构化扰动 $\Delta\theta$
\item $\mathcal{V}$:验证过程(Verification),从测试集上的观测特征推断模型是否包含水印
\end{itemize}

\textbf{关键认识}:与以往假设可以直接控制路由器输出的方法不同,我们认识到:
\begin{itemize}
\item \textbf{路由器是一个习得的函数}:$\mathbf{r} = \text{Router}(x) = \text{Softmax}(\mathbf{w}^T x + b)$,我们无法直接控制其输出
\item \textbf{编码器的真正任务}:不是直接控制 $\mathcal{S}(x)$,而是通过参数扰动 $\Delta\theta$ 在数据分布 $p(x)$ 上使 $\mathcal{S}(x)$ 的统计分布 $p(\mathcal{S}|\theta_{\text{wm}})$ 发生可检测的、偏向消息 $m$ 的偏移
\item \textbf{信道模型}:
  \begin{itemize}
  \item \textbf{输入}:参数扰动 $\Delta\theta$(而非消息 $m$ 或路由状态 $\mathcal{S}(x)$)
  \item \textbf{信道}:数据分布 $p(x)$ 对 $\Delta\theta$ 的"探测"和"响应",同一个 $\Delta\theta$ 在不同输入 $x_i$ 上产生不同的 $\mathcal{S}(x_i)$
  \item \textbf{输出}:验证者在测试集 $\mathcal{T} = \{x_1, \ldots, x_N\}$ 上观测到的特征统计量 $\mathbf{f} = [\mathcal{F}(x_1), \ldots, \mathcal{F}(x_N)]$
  \end{itemize}
\end{itemize}

%-------------------------------------------------------------------------------
\section{水印信息的编码空间}
%-------------------------------------------------------------------------------

\subsection{路由状态空间}

对于输入 $x$,在第 $l$ 层的路由状态表示为:
$$\mathcal{S}(x) = (\mathbf{r}, \Sigma, \pi) \in \Delta^{n-1} \times 2^{[n]} \times S_k$$

其中:
\begin{itemize}
\item $\mathbf{r} \in \Delta^{n-1}$:路由分布向量(位于概率单纯形)
\item $\Sigma \subset [n]$:激活的专家集合,满足 $|\Sigma| = k$(top-$k$ 稀疏门控)
\item $\pi \in S_k$:激活顺序(排列),表示 $\Sigma$ 中专家按路由权重降序排列
\end{itemize}

为避免记号歧义,我们明确区分:$\text{Softmax}(\cdot)$ 表示函数,$\Sigma$ 表示激活集合,$\pi$ 表示排列。

\subsection{可用信息维度}

路由状态空间提供了三个主要的信息维度用于编码水印:

\textbf{维度1:激活模式(组合码)}

从 $n$ 个专家中选择 $k$ 个激活,信息容量为:
$$I_{\text{pattern}} = \log_2 \binom{n}{k} = \log_2 \frac{n!}{k!(n-k)!}$$

对于 $n=128, k=2$ 的典型配置,使用斯特林近似(Stirling's approximation):
$$I_{\text{pattern}} \approx 13 \text{ bits}$$

\textbf{维度2:排列顺序(排列码)}

对前 $k$ 个专家的排列,信息容量为:
$$I_{\text{order}} = \log_2(k!)$$

对于 $k=4$:
$$I_{\text{order}} = \log_2(24) \approx 4.6 \text{ bits}$$

\textbf{注意}:该容量仅在"前 $k$ 专家可完全排序且可控"的假设下成立。若路由器含温度缩放或分数权重相近导致排序不稳定,应引入有限分辨率/随机噪声的"有效容量"修正。

\textbf{维度3:权重量化(连续码)}

由于 $\mathbf{r} \in \Delta^{n-1}$ 且满足 $\sum_i r_i = 1$,自由度为 $(n-1)$。在 top-$k$ 稀疏门控下,仅对激活的 $k$ 个专家权重进行量化,且满足 $\sum_{i \in \Sigma} r_i = 1$,因此自由度为 $(k-1)$。假设每个路由权重 $r_i$($i \in \Sigma$)量化为 $b$-bit 精度:
$$I_{\text{weight}} \leq (k-1) \cdot b \text{ bits}$$

对于 $k=4, b=4$(16级量化):
$$I_{\text{weight}} \leq 12 \text{ bits}$$

考虑温度 $T$ 与归一化噪声 $\sigma$ 的影响,有效容量需乘以系数 $\eta(T, \sigma) \in (0,1)$:
$$I_{\text{weight}}^{\text{eff}} = \eta(T, \sigma) \cdot (k-1) \cdot b$$

\textbf{总容量上界}:
\begin{align}
C_{\max} &= I_{\text{pattern}} + I_{\text{order}} + I_{\text{weight}}^{\text{eff}} \nonumber \\
&= \log_2 \binom{n}{k} + \log_2(k!) + \eta(T, \sigma) \cdot (k-1) \cdot b
\end{align}

%-------------------------------------------------------------------------------
\section{信息容量的量化}
%-------------------------------------------------------------------------------

\subsection{可实现容量(Achievable Capacity)与编码本视角}

从编码本(Codebook)视角,编码器 $\mathcal{E}$ 将水印消息 $m$ 映射为模型参数的结构化扰动 $\Delta\theta$。信道模型为"参数扰动 $\Delta\theta$ → 数据分布 $p(x)$ 的响应 → 路由状态的后验分布 $p(\mathcal{S}(x)|\theta_{\text{wm}})$ → 验证统计量 $\mathbf{f}$"。

在约束模型性能和隐蔽性的条件下,可实现的容量定义为:
$$C_{\text{achievable}} = \max_{p(m), p(\Delta\theta|m)} I(m; \mathbf{f})$$

其中 $\mathbf{f} = [\mathcal{F}(x_1), \ldots, \mathcal{F}(x_N)]$ 为测试集 $\mathcal{T}$ 上的特征统计量,$I(\cdot; \cdot)$ 为互信息。

\textbf{特征选择}:根据鲁棒性分析,排列码 $I_{\text{order}}$ 和权重量化 $I_{\text{weight}}$ 在攻击下几乎无法幸存。因此,鲁棒的验证者 $\mathcal{V}$ 应该只依赖于激活模式 $\Sigma$。信道输出 $\mathbf{f}$ 可以被精简为:在 $N$ 个样本上观测到的\textbf{激活模式 $\Sigma$ 的经验分布(或计数向量)}。

\subsection{率失真(Rate-Distortion)问题}

与以往针对单个输入 $x$ 的路由分布 $\mathbf{r}$ 定义失真的方法不同,我们必须在数据分布 $p(x)$ 上定义失真。率失真函数被重新定义为:

\begin{align}
R(D_{\text{perf}}, D_{\text{detect}}) &= \max_{p(m), p(\Delta\theta|m)} I(m; \mathbf{f}) \nonumber \\
&\quad \text{s.t.} \quad D_{\text{perf}} \leq \epsilon_1, \quad D_{\text{detect}} \leq \epsilon_2
\end{align}

其中失真是一个\textbf{二维向量} $D = (D_{\text{perf}}, D_{\text{detect}})$:

\begin{enumerate}
\item \textbf{性能失真 $D_{\text{perf}}$}:嵌入水印对模型主要任务造成的损害
$$D_{\text{perf}} = \Delta_{\text{perf}} = \text{Acc}_{\text{clean}} - \text{Acc}_{\text{watermarked}}$$

\item \textbf{统计失真 $D_{\text{detect}}$(可隐蔽性)}:水印在路由统计上留下的痕迹,可被量化为干净模型和水印模型在路由输出分布上的 KL 散度
$$D_{\text{detect}} = \mathbb{E}_{x \sim p(x)} \left[D_{\text{KL}}\left(p(\mathcal{S}(x)|\theta_{\text{clean}}) || p(\mathcal{S}(x)|\theta_{\text{wm}})\right)\right]$$
\end{enumerate}

\textbf{失真的双重作用}:
\begin{itemize}
\item $D_{\text{detect}}$ 越小,水印越\textbf{隐蔽}(越难被攻击者发现)
\item $D_{\text{detect}}$ 越大,水印越\textbf{可检测}(越容易被验证者验证)
\end{itemize}

\textbf{信息-性能-隐蔽性权衡}:通过扫描超参数(如 $\lambda_{\text{wm}}$),可以凭经验绘制出 $R(D_{\text{perf}}, D_{\text{detect}})$ 的帕累托前沿,其中:
\begin{itemize}
\item X轴:$D_{\text{perf}}$(模型准确率下降)
\item Y轴:$R$(水印检测器的统计显著性或 AUC)
\item Z轴:$D_{\text{detect}}$(隐蔽性,越小越好)
\end{itemize}

%-------------------------------------------------------------------------------
\section{基于约束优化的编码器算法}
%-------------------------------------------------------------------------------

\subsection{问题的形式化定义}

编码器 $\mathcal{E}$ 的任务是将水印消息 $m$ 映射为模型参数扰动 $\Delta\theta$。这可以形式化为一个约束优化问题:

\begin{align}
\max_{\Delta\theta} \quad &\underbrace{\mathbb{E}_{x \sim p(x)} \left[\log p(\Sigma_m | x, \theta_{\text{clean}} + \Delta\theta)\right]}_{\text{水印强度 (R)}} \nonumber \\
&\quad - \lambda_1 \underbrace{\Delta\mathcal{L}_{\text{task}}(\Delta\theta)}_{\text{性能失真 ($D_{\text{perf}}$)}} - \lambda_2 \underbrace{D_{\text{detect}}(\Delta\theta)}_{\text{隐蔽性失真 ($D_{\text{detect}}$)}}
\end{align}

其中:
\begin{itemize}
\item $m \rightarrow \Sigma_m$:通过带密钥 $K$ 的哈希函数 $H_K(m)$ 将消息 $m$ 映射到目标激活模式 $\Sigma_m$(例如,对于 $n=8, k=2$,共有 $\binom{8}{2} = 28$ 种可能的激活模式)
\item $\lambda_1, \lambda_2$:拉格朗日乘子,用于在 $R$、$D_{\text{perf}}$ 和 $D_{\text{detect}}$ 之间进行权衡,构成帕累托前沿
\end{itemize}

\subsection{实用的优化算法}

我们不需要从头开始求解上述问题,可以将其构建为一个\textbf{带约束的微调(Fine-tuning)过程}:

\textbf{算法:基于约束优化的水印嵌入}

\begin{enumerate}
\item \textbf{目标定义}:选择消息 $m$,通过哈希函数 $H_K(m)$ 生成目标激活模式 $\Sigma_m$。

\item \textbf{损失函数设计}:定义复合损失函数 $\mathcal{L}_{\text{total}}$ 来微调 $\theta_{\text{clean}}$:
$$\mathcal{L}_{\text{total}} = \mathcal{L}_{\text{task}} + \lambda_{\text{wm}} \cdot \mathcal{L}_{\text{wm}}$$

其中:
\begin{itemize}
\item $\mathcal{L}_{\text{task}}$:原始模型任务损失(如交叉熵),确保保持在 $D_{\text{perf}} \leq \epsilon_1$ 的约束内
\item $\lambda_{\text{wm}}$:水印强度超参数,直接控制 $R$ 和 $D_{\text{perf}}$ 之间的权衡
\item $\mathcal{L}_{\text{wm}}$:水印损失,用于实现 $\max p(\Sigma_m)$ 的目标
\end{itemize}

\item \textbf{水印损失 $\mathcal{L}_{\text{wm}}$ 的设计}:对于一个输入 $x$,路由器产生 $n$ 个 logits:$\mathbf{l} = [l_1, \ldots, l_n]$。我们的目标是使目标模式 $\Sigma_m$ 中的专家成为 Top-$k$。采用\textbf{基于间隔(Margin-based)的损失}:

\begin{itemize}
\item 找到 $\Sigma_m$ 中的最低分:$l_{\min\_\text{target}} = \min_{i \in \Sigma_m} l_i$
\item 找到\textbf{非} $\Sigma_m$ 中的最高分:$l_{\max\_\text{other}} = \max_{i \notin \Sigma_m} l_i$
\item \textbf{损失函数}:$\mathcal{L}_{\text{wm}} = \text{ReLU}(l_{\max\_\text{other}} - l_{\min\_\text{target}} + \text{margin})$
\end{itemize}

该损失的含义是:"要求目标专家($\Sigma_m$ 中)的最低分,必须比所有其他专家的最高分还要高出一个 $\text{margin}$。如果这个条件满足,损失为0;否则,就施加一个惩罚。"

\item \textbf{执行编码}:使用 $\mathcal{L}_{\text{total}}$ 对 $\theta_{\text{clean}}$(或仅仅是路由器参数)进行微调几个(甚至几十个)epoch。

这个过程\textbf{自动}找到了一个"折衷"的 $\Delta\theta$:
\begin{itemize}
\item 它只在那些\textbf{不会}严重损害 $\mathcal{L}_{\text{task}}$ 的输入 $x$ 上"悄悄地"推高 $\Sigma_m$ 的概率
\item 它在"最容易"被操纵的输入上嵌入水印,而不是在所有输入上强行嵌入
\item $\lambda_{\text{wm}}$ 的大小直接控制了 $R$ 和 $D_{\text{perf}}$ 之间的权衡
\end{itemize}
\end{enumerate}

\subsection{与理论框架的连接}

该算法完美地连接回了形式化框架:

\begin{itemize}
\item \textbf{编码器 $\mathcal{E}$}:不再是简单的加法 $\tilde{\mathbf{w}} = \mathbf{w} + \delta \mathbf{b}_{\text{target}}$,而是一个\textbf{优化过程}(即上述的微调算法)

\item \textbf{可实现容量 $C_{\text{achievable}}$}:容量 $R$ 不再是理论上的 $\log_2 \binom{n}{k}$,而是由 $\lambda_{\text{wm}}$ 控制的、在 $D_{\text{perf}}$ 和 $D_{\text{detect}}$ 约束下的\textbf{实际}互信息 $I(m; \mathbf{f})$

\item \textbf{率失真 $R(D)$}:可以通过\textbf{扫描 $\lambda_{\text{wm}}$} 来凭经验绘制出 $R(D_{\text{perf}}, D_{\text{detect}})$ 曲线,完美体现了"信息-性能-隐蔽性权衡"的思想
\end{itemize}

%-------------------------------------------------------------------------------
\section{鲁棒性的形式化定义}
%-------------------------------------------------------------------------------

\subsection{对抗性攻击下的鲁棒性(Min-Max 定义)}

设攻击为随机变换 $\mathcal{A}: \mathcal{S} \to \mathcal{S}'$,攻击族 $\mathcal{A}$ 包含:
\begin{itemize}
\item 微调(fine-tune):路由器单独/联合微调
\item 蒸馏(distill):蒸馏到无水印模型
\item 专家重排/剪枝(expert reindex/prune)
\item 温度/门控策略变更:top-$k$ → top-1/2 切换
\item 输入域漂移:输入 Noising/域外测试集
\item 量化/低比特化
\item LoRA 注入、梯度手术(冻结路由层)等
\end{itemize}

\textbf{鲁棒性定义(极小极大准则)}:
$$\mathcal{R} = \inf_{\mathcal{A} \in \mathcal{A}} \Pr\big[\text{Verify}(f_{\text{watermarked}}, \mathcal{A}) = 1\big]$$

其中 $\text{Verify}(f, \mathcal{A}) = 1$ 表示在攻击 $\mathcal{A}$ 后仍能成功验证。我们区分\textbf{随机对手}(从 $\mathcal{A}$ 中随机选择攻击)与\textbf{自适应对手}(根据验证器策略选择最优攻击)。

\subsection{鲁棒性的下界(基于 Le Cam/Pinsker 方法)}

对于独立同分布的攻击,水印的鲁棒性下界基于\textbf{Le Cam 两点法}或\textbf{信息不等式(Pinsker/Van Trees)}给出可检性(distinguishability)下界。

设攻击后的路由状态为 $\mathcal{S}'(x) = \mathcal{A}(\mathcal{S}(x))$,水印消息空间大小为 $|\mathcal{M}|$。基于\textbf{Pinsker 不等式},总变分距离(Total Variation)与 KL 散度的关系为:
$$\text{TV}(p_w, p_c) \leq \sqrt{\frac{1}{2} D_{\text{KL}}(p_w || p_c)}$$
其中 $p_w = p_{\text{watermarked}}$ 和 $p_c = p_{\text{clean}}$ 分别为带水印和干净模型的路由分布。

\textbf{可检性下界}:若攻击后的分布满足 $D_{\text{KL}}(p_w || p_c) \geq \delta$,则验证成功率下界为:
$$\mathcal{R} \geq 1 - \exp\left(-\frac{\delta}{2}\right) - \alpha$$

其中 $\alpha$ 为假阳性率上界。该下界更贴近"验证器"的性质,而非直接使用 Fano 不等式。

\subsection{对特定攻击的鲁棒性量化}

\textbf{攻击1:路由器重训练}

设路由器参数的扰动为 $\Delta \mathbf{w}, \Delta b$,考虑 Kullback-Leibler 散度:
$$D_{\text{KL}}(\mathbf{r} || \mathbf{r}') = \sum_i r_i \log \frac{r_i}{r_i'} \leq \beta$$

鲁棒性与 $\beta$ 的关系:
$$\text{Rob}_{\text{retrain}} = 1 - \exp(-\lambda D_{\text{KL}})$$

其中 $\lambda$ 为编码的纠错码强度参数。

\textbf{攻击2:模型蒸馏}

设蒸馏温度为 $T$,学生模型的路由分布为 $\mathbf{r}_s$:
$$\text{Rob}_{\text{distill}} = \exp\left(-\frac{D_{\text{KL}}(\mathbf{r}^T || \mathbf{r}_s^T)}{H(\mathbf{r})}\right)$$

其中 $H(\mathbf{r})$ 为路由分布的熵。

\textbf{攻击3:量化与剪枝}

对于 $q$-bit 量化:
$$\text{Rob}_{\text{quant}} = \left(1 - \frac{2^{-q}}{2}\right)^{n \cdot I_{\text{weight}}}$$

%-------------------------------------------------------------------------------
\section{验证过程与检测能力}
%-------------------------------------------------------------------------------

\subsection{假设检验框架(复合假设)}

验证问题可形式化为\textbf{复合假设检验}:
\begin{itemize}
\item $H_0$:模型未被水印化(原始模型),参数 $\theta_0 \in \Theta_0$ 未知
\item $H_1$:模型被正确的水印化,参数 $\theta_1 \in \Theta_1$ 未知且受攻击扰动
\end{itemize}

由于参数未知且受攻击扰动,$H_0$ 和 $H_1$ 均为复合假设。给定测试集 $\mathcal{T} = \{x_1, \ldots, x_N\}$,提取特征向量:
$$\mathbf{f} = [\mathcal{F}(x_1), \ldots, \mathcal{F}(x_N)]$$

\textbf{特征选择}:根据鲁棒性分析,排列码 $I_{\text{order}}$ 和权重量化 $I_{\text{weight}}$ 在攻击下几乎无法幸存。因此,鲁棒的验证者 $\mathcal{V}$ 应该\textbf{只依赖于激活模式 $\Sigma$}。特征提取函数 $\mathcal{F}$ 定义为:
$$\mathcal{F}(x) = \Sigma(x)$$

即只提取激活的专家集合。特征向量为:
$$\mathbf{f} = (\Sigma_1, \ldots, \Sigma_N)$$

\textbf{信道输出}:验证者观测到的信道输出 $Y$ 是在 $N$ 个样本上观测到的\textbf{激活模式 $\Sigma$ 的经验分布(或计数向量)}。对于 $n$ 个专家、$k$ 个激活的配置,共有 $\binom{n}{k}$ 种可能的激活模式,因此 $\mathbf{f}$ 可以表示为长度为 $\binom{n}{k}$ 的计数向量,其中每个元素表示对应激活模式在测试集上的出现次数。

\subsection{广义似然比检验(GLRT)与序贯概率比检验(SPRT)}

由于 $H_0$ 和 $H_1$ 为复合假设,采用\textbf{广义似然比检验(GLRT)}以消去未知参数:
$$\Lambda_{\text{GLRT}}(\mathbf{f}) = \frac{\max_{\theta_1 \in \Theta_1} p(\mathbf{f}|\theta_1)}{\max_{\theta_0 \in \Theta_0} p(\mathbf{f}|\theta_0)} \gtrless \tau$$

或采用\textbf{序贯概率比检验(SPRT)}以降低测试样本开销:
$$\Lambda_{\text{SPRT}}(\mathbf{f}_t) = \prod_{i=1}^t \frac{p(\mathbf{f}_i|H_1)}{p(\mathbf{f}_i|H_0)} \gtrless \tau$$

其中 $t$ 为当前样本数,SPRT 可在达到决策阈值时提前终止。

\textbf{检测功率(功效)}定义为:
$$\beta = \inf_{\theta_1 \in \Theta_1} P(\Lambda > \tau | H_1, \theta_1)$$

\textbf{假阳性率}(False Positive Rate):
$$\alpha = \sup_{\theta_0 \in \Theta_0} P(\Lambda > \tau | H_0, \theta_0)$$

\subsection{ROC曲线与AUC下界(基于错判率上界)}

检测质量由接收者操作特征(ROC)曲线量化:
$$\text{AUC} = \int_0^1 \text{TPR}(\text{FPR}) \, d(\text{FPR})$$

对于信息论上界,我们采用\textbf{Chernoff/Bhattacharyya 上界}先给出\textbf{错判率}或\textbf{Bayes 风险}的指数型上界,再间接推导对 AUC 的保守下界。

\textbf{Bhattacharyya 距离}:
$$D_B = -\ln \int \sqrt{p(\mathbf{f}|H_0) p(\mathbf{f}|H_1)} \, d\mathbf{f}$$

对于高斯分布族,错判率上界为:
$$P_{\text{error}} \leq \exp(-D_B)$$

进而可推导 AUC 的保守下界(需注明适用分布族,如高斯族):
$$\text{AUC} \geq 1 - \exp(-D_B) - \alpha$$

其中 $\alpha$ 为假阳性率上界。

%-------------------------------------------------------------------------------
\section{综合性能指标}
%-------------------------------------------------------------------------------

\subsection{水印质量函数与帕累托前沿}

综合定义水印系统的质量为三维指标:
$$\mathcal{Q} = (C_{\text{achievable}}, \mathcal{R}, \text{AUC})$$

其中 $C_{\text{achievable}}$ 为可实现容量,$\mathcal{R}$ 为鲁棒性(min-max 定义),$\text{AUC}$ 为检测 AUC。

考虑多目标优化:
$$\max \{C_{\text{achievable}}, \mathcal{R}, \text{AUC}\}$$
$$\text{s.t.} \quad \Delta_{\text{perf}} \leq \epsilon, \quad \alpha \leq \alpha_{\max}$$

\textbf{帕累托前沿}定义为非被支配的解集合,即不存在其他解在所有目标上都不劣于当前解且至少在一个目标上更优。

\textbf{权重设定}:质量函数的加权形式 $Q_{\text{normalized}} = \alpha C_n + \beta R_n + \gamma A_n$(其中 $\alpha + \beta + \gamma = 1$)仅用于\textbf{政策选择}(根据具体应用需求选择折中点),而非理论结论。权重来源与任务依赖需明确:不同数据集/任务(如生成式 vs 分类式 MoE)可能需要不同的权重配置。建议以\textbf{Pareto 前沿}为主图呈现不加权的解集,再由应用需求选择折中点。

%-------------------------------------------------------------------------------
\section{数学推导示例}
%-------------------------------------------------------------------------------

\subsection{示例:基于激活模式的水印嵌入}

假设设计仅基于激活模式 $\Sigma$ 进行水印,使用 $n=8$ 个专家、$k=2$ 个激活的配置:

\textbf{信息来源}:从 $n=8$ 个专家中选择 $k=2$ 个激活,共有 $\binom{8}{2} = 28$ 种可能的激活模式
$$I_{\text{pattern}} = \log_2 \binom{8}{2} = \log_2 28 \approx 4.8 \text{ bits}$$

\textbf{消息映射}:通过带密钥 $K$ 的哈希函数 $H_K(m)$ 将消息 $m$(例如 "U-C-Berkeley")映射到目标激活模式 $\Sigma_m$(例如,$\Sigma_m = \{\text{专家 3, 专家 7}\}$)。

\textbf{编码方案}:使用基于约束优化的微调过程,而非简单的偏置向量加法。定义复合损失函数:
$$\mathcal{L}_{\text{total}} = \mathcal{L}_{\text{task}} + \lambda_{\text{wm}} \cdot \mathcal{L}_{\text{wm}}$$

其中水印损失 $\mathcal{L}_{\text{wm}}$ 采用基于间隔的损失:
$$\mathcal{L}_{\text{wm}} = \text{ReLU}\left(\max_{i \notin \Sigma_m} l_i - \min_{i \in \Sigma_m} l_i + \text{margin}\right)$$

该损失要求目标专家($\Sigma_m$ 中)的最低 logit 必须比所有其他专家的最高 logit 还要高出一个 $\text{margin}$。

\textbf{性能约束}:在数据分布 $p(x)$ 上的性能失真和统计失真
$$D_{\text{perf}} = \text{Acc}_{\text{clean}} - \text{Acc}_{\text{watermarked}} \leq \epsilon_1$$
$$D_{\text{detect}} = \mathbb{E}_{x \sim p(x)} \left[D_{\text{KL}}\left(p(\mathcal{S}(x)|\theta_{\text{clean}}) || p(\mathcal{S}(x)|\theta_{\text{wm}})\right)\right] \leq \epsilon_2$$

\textbf{可实现容量}:通过扫描 $\lambda_{\text{wm}}$,可以凭经验测量实际互信息 $I(m; \mathbf{f})$,其中 $\mathbf{f}$ 是测试集上激活模式的经验分布。实际容量 $C_{\text{achievable}}$ 通常小于理论容量 $I_{\text{pattern}}$,因为需要在性能失真和统计失真之间进行权衡。

\textbf{鲁棒性下界}:对路由器重训练攻击,基于 Pinsker 不等式,若攻击后的分布满足 $D_{\text{KL}}(p_{\text{wm}} || p_{\text{clean}}) \geq \delta$,则验证成功率下界为:
$$P_{\text{detect}} \geq 1 - \exp\left(-\frac{\delta}{2}\right) - \alpha$$

其中 $\alpha$ 为假阳性率上界。

%-------------------------------------------------------------------------------
\section{开放的理论问题}
%-------------------------------------------------------------------------------

尽管本文建立了 MoE 路由水印的信息论框架,但仍存在一些开放的理论问题:

\begin{enumerate}
\item \textbf{稀疏门控与单纯形容量的精确化}:在 top-$k$ 稀疏门控与概率单纯形约束下,路由水印的有效自由度与容量是多少?如何用类型法(method of types)为离散部分给出可靠编码区与错误指数?
\item \textbf{嵌入-性能的率失真闭环}:在给定性能降级门限下,能否得到显式的容量上/下界?如何将性能损失上界映射为对路由分布的散度球约束(KL 或 Rényi 散度)?
\item \textbf{复合假设检验的稳健验证器}:如何构造稳健最优的验证器以处理攻击与域漂移?如何比较以 $\Sigma$、$\pi$、$\mathbf{r}_\Sigma$ 为特征的不同充分性与功效?
\item \textbf{隐蔽性(Steganographic Security)与可检性下界}:路由分布的统计隐蔽性如何量化?如何定义对手的检测器族并给出最小可辨性下界?
\item \textbf{联合攻击}:多种攻击同时进行时的鲁棒性如何分析?是否存在攻击之间的协同效应?
\end{enumerate}

%-------------------------------------------------------------------------------
\section{结论}
%-------------------------------------------------------------------------------

本文提出了 MoE 路由水印的信息论形式化框架,为路由水印的设计和评估提供了严格的理论基础。与以往假设可直接控制路由器输出的方法不同,我们建立了基于参数扰动的信道模型,明确了编码器、信道和输出的定义。我们量化了激活模式(组合码)的编码容量,这是最鲁棒的信息维度,而排列码和权重量化在攻击下几乎无法幸存。我们建立了二维率失真框架,同时考虑性能失真 $D_{\text{perf}}$ 和统计失真 $D_{\text{detect}}$(隐蔽性),并提出了基于约束优化的编码器算法,通过带约束的微调过程实现水印嵌入。采用极小极大准则,我们定义了鲁棒性并基于 Le Cam/Pinsker 方法推导了可检性下界。在验证过程中,我们构建了基于复合假设检验的框架(GLRT/SPRT),强调只使用激活模式特征进行验证,并基于错判率上界间接推导了检测质量的 AUC 下界。这些理论结果为 MoE 路由水印的实践应用提供了指导,并为未来的研究指明了方向,包括稀疏门控与单纯形容量的精确化、嵌入-性能-隐蔽性的率失真闭环、复合假设检验的稳健验证器以及隐蔽性分析等开放问题。

%%%%%%%%%%%%%%%%%%%%%%%%%%%%%%%%%%%%%%%%%%%%%%%%%%%%%%%%%%%%%%%%%%%%%%%%%%%%%%%%
\end{document}
%%%%%%%%%%%%%%%%%%%%%%%%%%%%%%%%%%%%%%%%%%%%%%%%%%%%%%%%%%%%%%%%%%%%%%%%%%%%%%%%

